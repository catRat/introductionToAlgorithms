\chapter{算法基础}
本章将介绍一个贯穿本书的框架,后续的算法设计与分析都是在这个框架中进行的。这一部分内容基本上是独立的,但也有对第3章和第4章中一些内容的引用。
\par 首先,我们考察求解第1章中引入的排序问题的插入排序算法。我们定义一种对已经编写过计算机程序的读者来说应该熟悉的“伪代码”,并用它来表示我们将如何说明算法。然后,在说明了插入排序算法后,我们将证明该算法能正确地排序并分析其运行时间。这种分析引入了一种记号,该记号关注时间如何随着将被排序的项数而增加。在讨论完插入排序之后,我们引入用于算法设计的分而治之,并使用这种方法开发一个称谓归并排序的算法。最后,我们分析归并排序的运行时间。
\section{插入算法}
我们的第一个算法求解第1章引入的排序算法: 
\par 输入:$n$个数的一个序列$<a_1,a_2, \cdots, a_n>$。
\par 输出:输入序列的一个排序$<a'_1, a'_2, \cdots,a'_n>$。
\par 我们排序的数也称为关键词。虽然概念上我们在排序一个序列,但是输入是以$n$个元素的数组的形式出现的。
\par 伪代码与真代码的另一个区别是伪代码通常不关心软件工程的问题。为了更简洁地表达算法的本质,常常忽略数据抽象、模块性和错误处理的问题。
\begin{verbatim}
INSERTION-SORT(A)
  for j = 2 to A.length
    key = A[j]
    //Insert A[j] into the sorted sequence A[1..j -1].
    i = j -1
    while i > 0 and A[i] > key
      A[i + 1] = A[i]
      i = i - 1
    A[i + 1] = key
\end{verbatim}